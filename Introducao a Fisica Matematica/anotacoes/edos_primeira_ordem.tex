\section{EDOs homogêneas, inomogêneas - fator integrante}
Para equações homogêneas e inomogêneas lineares com variável
independente \( t \) e dependente \( y \), queremos saber a
cara de \( y(t) \) dada na equação \( a(t) y'(t) + b(t) y(t) = c(t) \) 

onde \( a, b  \) e \( c \) são funções dadas de \( t \) .


\paragraph{$\blacksquare$ Exemplo}: \( t^2 y' + 5t y = t^3 - 1 \)
Se \( c(t) = 0 \) então a EDO é homogênea e caso contrário inomogênea.

\begin{itemize}
  \item[$\blacksquare$] Caso homogêneo: 
  \[ a(t) y'(t) + b(t) y(t) = 0 \]
  \[ y'(t) + p(t) y(t) = 0, p(t) = \frac{b(t)}{a(t)} \]
  \[ \Leftrightarrow \frac{y'(t)}{y(t)} = - p(t) \Leftrightarrow \frac{1}{y} \frac{dy}{dt} = -p(t) \]
  \[ \Leftrightarrow \int \frac{1}{y} dy = - \int p(t) dt \]
  \[ \Leftrightarrow  \int \frac{dy}{y} = - \int p(t) dt  \]
  \[ \Leftrightarrow  \ln |y| = - \int p(t) dt \]
  \[ \Leftrightarrow  |y| = e^{- \int p(t) dt + \mathbb{C}} = e^{-\int p(t) dt} e^{\mathbb{C}} \]
  \[ \Leftrightarrow  |y| = |A|e^{- \int p(t) dt}, |A| = e^{\mathbb{C}} \]
  \[ \Leftrightarrow  y(t) = Ae^{- \int p(t) dt}\]

  O valor de $A$ vai depender das condições iniciais do problema.
  
  \item[$\blacksquare$] Caso inomogêneo: \textbf{fator integrante}

  \[ a(t) y'(t) + b(t) y(t) = c(t) \]
  \[ y'(t) + p(t) y(t) = r(t), p(t) = \frac{b(t)}{a(t)}, r(t) = \frac{c(t)}{a(t)} \]

  A estratégia para resolver com fator integrante é chegar à uma equação
  que permita-nos escrever o lado esquerdo da equação usando a regra do produto
  para derivadas, ou seja,
  \[ \frac{d  }{d x} \left( f g \right) = g \frac{d}{dx}f  + f \frac{d}{dx}g \]

  Para tal, precisamos de uma combinação linear do tipo
  \( \mu (t) y'(t) + \mu (t) p(t) y(t) = \mu (t) r(t) \), onde \( \mu(t) \) é o fator
  integrante.
  A relação
  \[ \frac{d}{dt} \left(\mu(t) y(t) \right) = \mu(t) y'(t) + \mu(t) p(t) y(t) \] 

  é válida so e se só \( \mu(t) \) obedece a relação

  \begin{equation}
    \frac{d}{dt} \mu(t) = p(t) \mu(t) 
    \label{eq:fator_integrante}
  \end{equation}
\end{itemize}


\paragraph{$\blacksquare$ Exemplo}: \( t^2 y'(t) + 5ty(t) = t^3 - 1 \) 

\[ \Leftrightarrow y' + \frac{5}{t}y = t - \frac{1}{t^2}  \]
\[ \Leftrightarrow \frac{d}{dt}\mu = \mu y' + \frac{5}{t}\mu y = \left( t - \frac{1}{t^2}\right)\mu \]
\[ \Leftrightarrow  \frac{d\mu}{dt}= \frac{5\mu}{t} \]
Essa relação é uma EDO homogênea simples, então temos
\[ \Leftrightarrow \frac{\mu'}{\mu} = \frac{5}{t} \]
\[ \Leftrightarrow \ln|\mu| = 5 \ln |t| \Rightarrow \mu = t^5\]

voltando à equação anterior temos

\[ \frac{d}{dt} \left( t^5 y \right) = t^6 - t^3 \]
\[ t^5 y = \frac{t^7}{7} - \frac{t^4}{4} + \mathbb{C} \]

\[ \Rightarrow y(t) = \frac{t^2}{7}- \frac{1}{4t} + \frac{\mathbb{C}}{t^5} \]


\section{Método de variação de paramêtros}
A solução de uma EDO linear inomogênea de primeira ordem é a soma da solução geral da homogênea
associada com uma solução particular \textbf{qualquer} da EDO inomogênea original.
Podemos portanto `` construir '' a solução de uma EDO inomogênea.


\[ y' + p(t) y = r(t) \rightarrow y(t) = y_{H} (t) + y_{P} (t) \]
onde \( y_{H} \) é a solução da EDO homogênea associada, \( y_P \) é a paramétrica obtida a partir
da homogênea e \( y \) é a solução geral da EDO dada.

\paragraph{$\blacksquare$ Exemplo} \( t y' - t^2 = y, t >0 \) 

\[ t y' - y = t^2 \Rightarrow y' - \frac{1}{t}y = t  \]
Fator integrante \( \mu y' - \frac{\mu}{t}y = \mu t \Rightarrow \frac{d}{dt} \left( \mu y \right) = \mu t \) 
\[ \mu ' = - \frac{\mu }{t} \Leftrightarrow \frac{\mu'}{\mu} = - \frac{1}{t} \Rightarrow \frac{d\mu}{\mu}= - \frac{dt}{t} \Leftrightarrow \ln|\mu| = - \ln|t|\]
\[ \Rightarrow \mu = \frac{A}{t} \]

\[\Rightarrow \frac{d}{dt} \left( \frac{y}{t} \right)  = 1 \Leftrightarrow \frac{1}{t} y = 1 \Leftrightarrow \frac{1}{t}y = t +
  \mathbb{C} \]
\[ \Leftrightarrow y(t) = t^2 + \mathbb{C} t \]

Para \( y(1) = 0 \Rightarrow 0 = 1^2 + \mathbb{C} \cdot 1 \Rightarrow \mathbb{C} = -1 \Leftrightarrow y(t) = t^2 - t \) 

Checando a propriedade da variação de paramêtros a partir desse resultado.
Chutando \( y_P (t) = t^2 \) a solução geral da homogênea: \( y_{H}' - \frac{1}{t} y_H = 0 \) 
\[ \frac{y'_{H}}{y_{H}} = \frac{1}{t} \Leftrightarrow \frac{d y_{H}}{y_{H}} = \frac{dt}{t} \Leftrightarrow \ln|y_H| = \ln t + \mathbb{C}  \]
\[  |y_H(t)| = e^{\mathbb{C}} t \Leftrightarrow y_H(t) = \mathbb{C} t \]
\[ \Leftrightarrow y(t) = \mathbb{C} t + t^2  \]

\section{EDOs separáveis}
Uma EDO de primeira ordem é dita ser separável se pode ser colocada na forma

\begin{equation}
  f(y) y' = g(t) \leftrightarrow M(y)y' + N(t) = 0 \Leftrightarrow M(y) dy + N(t) dt = 0
  \label{eq:edo_separavel}
\end{equation}

\section{Modelagem de sistemas usando EDOs}
Tratamos as populações a partir de densidades para evitar trabalhar com sistemas discretos que podem
levar a descontinuidades. Na realidade quando implementamos alguma simulação podemos tentar
considerar as discretizações do sistema. Na teoria estamos tratando dos modelos mais simples
possíveis.
\subsection{EDOs autonômas e crescimento populacional}
Sejam
\begin{itemize}
 \item[$\bullet$] y: densidade de ``entes'': Substâncias, presas $\times$ predadores, moleculas...
 \item[$\bullet$] t: tempo
\end{itemize}

Supondo \( \frac{dy}{dt} \propto y \Leftrightarrow \frac{dy}{dt}=  r y, r > 0 \) , onde \( r \) é a taxa de
crescimento. Esse é o modelo bem simplificado de um sistema fechado.

Temos
\[ \frac{dy}{y}= r dt \Leftrightarrow \ln | y| = rt + \mathbb{C} \Leftrightarrow y(t) = A e^{rt}\]
supondo \( y(t_0) = y_0 \Rightarrow y_0 = A e^{rt_0} \Rightarrow A = y_0 e^{-rt_0} \), logo
\[ y(t) = y_0 e^{r(t-t_0)} \]
note que para qualquer \( t \) a solução é apenas transladada.

\subsection{Equação logística}
Leva em consideração um limite superior para a solução. Isso é mais ``realista'' já que tem um fator
limitante como população cujo crescimento é proporcional.

\[ \frac{dy}{dt}= r(y) y = r_0 \left( 1 - \frac{y}{k} \right)y, r(y) \text{ é decrescente. }\]

[IMAGEM] 

Ponto \( k \) (valor crítico): a partir desse ponto a taxa de ``mortalidade'' é menor que a
de ``natalidade'', ou seja, a população decresce.

\[ f(y) = \frac{dy}{dt} = r_0 \left( 1 - \frac{y}{k} \right)y \]

[IMAGEM] [IMAGEM]

\subsection{Resolvendo a equação logística}
\[ \frac{dy}{dt} = r_0 \left( 1 - \frac{y}{k} \right)y  \Leftrightarrow y(t) = 0, y(t) = k \]
são soluções, logo \( \frac{dy}{y \left( 1 - \frac{y}{k} \right)}  = r_0 dt \)
integrar essa relação nos leva à uma integral por fração parcial. A forma de resolver
isso está no apendice [citar aqui]

resolvendo a integral temos

\[ \ln \bigg| \frac{y}{y- k} \bigg | = r_0 t + \mathbb{C} \]
\[ \frac{y}{y-k} = A e^{r_0 t} \Rightarrow y = A e^{r_0 t} (y - k) \]
\[ y \left( A e^{r_0 t} \right) = - k e^{r_0 t} A \]
\[ \Rightarrow y(t) = \frac{K}{-A^{-1} e^{-r_0 t} + 1} \]

\section{EDOs exatas}

Partindo das EDOS separáveis, temos \( f(y)dy + g(t) dt = 0 \),
ou seja \( F(y) + G(t) = \text{cte}, \frac{dF}{dy} = f(y), \frac{dG}{dt} = g(t) \) 

\paragraph{Generalizando}: supondo que existem um classe de EDOS cujas soluções podem ser dadsa por
 \( \psi (t, y(t)) = \text{cte} \), onde \( \psi(t, y(t)) = F(t, y(t)) + G(t, y(t)) \). A função
 \( \psi \) é um lugar geométrico em \( x, y \), uma curva de nível no plano de forma geral
 \( x = t \)  e \( y = y(t) \), para \( \psi(x, y) \).
 \[ \left( \frac{\partial }{\partial x} \psi \right) \frac{dx}{dt} + \left( \frac{\partial }{\partial y} \psi \right)\frac{d y }{d t} = 0 \]

 Com \( \psi_{, x} = \frac{\partial }{\partial x} \psi, \psi_{, y} = \frac{\partial }{\partial y} \psi\), substituindo na equação acima temos
 \[ \psi_{,x} + \psi_{,y}y' = 0 \]
 Isso é uma EDO separáel e pode ser escrita como
 \[ M(x, y) + N(x, y) y'(x) = 0 \]
 onde \( M(x, y) = \frac{\partial }{\partial x} \psi \)  e \( N(x, y) = \frac{\partial }{\partial y} \psi \), portanto
 \[ \psi(x, y(x)) = \text{cte} \]
 
 Qualquer EDO que pode ser escrita dessa forma é chamada de EDO exata.



 \paragraph{$\blacksquare$} Explicação (rasa) sobre essa manipulação:

 \[ \nabla \psi = \left( \frac{\partial }{\partial x} \psi, \frac{\partial }{\partial y} \psi \right) \]
 identificação de \( \frac{\partial }{\partial x} \psi \)  e \( \frac{\partial }{\partial y} \psi \), vale que
 \( \nabla \times \psi = 0 \Leftrightarrow \nabla \times \psi = \frac{\partial }{\partial y} \psi - \frac{\partial }{\partial x} \psi = 0  \Leftrightarrow \frac{\partial M}{\partial y} = \frac{\partial N}{\partial x}\).
 Essa é a condição necessária (e nesse caso suficiente) para encontrar a função \( \psi \).
 



    \subsection{Caso base}
    \paragraph{$\blacksquare$ Exemplo}: $(\sin x + x^2 e^y -1) y' + (y \cos x + 2x e^y ) = 0$ 
    Verificar validade de \( M(x, y) + N(x, y) y' = 0 \), com \( M_{,x} \overset{?}{=} N_{,y} \) 

    \[ M(x, y) = y \cos x + 2x e^y \mapsto \frac{\partial }{\partial y} M = \cos x + 2x e^y \]
    \[ N(x, y) = \sin x + x^2 e^y - 1 \mapsto \frac{\partial }{\partial x} N = \cos x + ex e^y \]

    \( M_{,x} = N_{,y} \), a equação é exata.


    Encontrando a função \( \psi(x, y) \) 

    \[ \frac{\partial \psi}{\partial x}  = y \cos x + 2x e^y \Rightarrow \psi(x, y) = y \sin x + x^2 e^y + h(y) \]
    fazendo \( \frac{\partial }{\partial y} \psi(x, y) = \frac{\partial }{\partial y}  \left( y \sin x + x^2 e^y + h(y) \right) \) 
    \[ \frac{\partial }{\partial y} \psi(x, y) = \sin x + x^2 e^y + h'(y) = \sin x + x^2 e^y - 1 \]
    \[ \Leftrightarrow h'(y) = -1 \Leftrightarrow h(y) = -y + \mathbb{C} \]
    \[ \Longrightarrow  \psi(x, y) = y \sin x + x^2 e^y - y + \mathbb{C} \]

    Portanto, a solução é dada pelo \( y(x) \) que satistfaz \( y(x) \sin x + x^2 e^{y(x)} - y(x) = \text{cte} \).



    \subsection{Caso que precisa de fator integrante}
    
 
    \paragraph{$\blacksquare$ Exemplo}: equação não exata e fator integrante para equações não exatas
    \[(2 x + 4y) + (2x - 2y) y' = 0\]


    Verificar a validade \( M(x, y) + N(x, y) y' = 0 \):
    \[ \begin{cases}
         M = 2x + 4y \\
         N = 2x - 2y
       \end{cases} \Rightarrow \begin{cases}
                       M_{, y} = 4 \\
                       N_{, x} = 2
                     \end{cases} \Rightarrow \text{Não satisfazem a condição} \]

     Para resolver usamos fator integrante: supondo que \( M(x, y) + N(x, y)y' = 0 \) é não
     exata. Encontrar \( \mu = \mu(x, y) \), tal que \( \mu M + \mu N y' = 0 \) seja exata.

     \[ \widetilde{M} = \mu M , \widetilde{N} = \mu N \]
     então \( \widetilde{M} + \widetilde{N} y' = 0 \), logo
     \[ (\mu M)_{, y} = (\mu N)_{,x} \]
   



     \paragraph{$\blacksquare$ Exemplo}: $ \left( 2 \sin(y) - x \right) y' = \tan(y)$
     \[ M(x, y) = - \tan(y) \Rightarrow \frac{\partial }{\partial y} M(x, y) =  -  \frac{\partial }{\partial y} \left(
         \frac{\sin(y)}{\cos(y)} \right) = - \frac{\cos (y)}{\cos(y)} - \frac{\sin^2 (y)}{\cos^2
         (y)}= - 1 - \tan^2 (y) = - \sec^2 (y) \]
    

     \[ N(x, y) = 2 \sin(y) - x \Rightarrow \frac{\partial }{\partial x}  N(x, y) = -1  \]
     
     \[ N_x  \neq  M_y \]
     
     Logo, pela forma que foi dada não é uma equação exata, precisamos encontrar algum fator
     integrante para resolve-la.
     
     Nessa situação chutamos primeiro fatores integrantes do tipo \( \mu(x, y) \) e outras variações
     que envolvam as duas variáveis ou uma delas, como \( \mu(x) \), \( \mu(y) \), \( \mu(x/y) \), \(
     \mu(xy) \), \( [\reflectbox{$\dots$}] \). A partir do fator integrante substituimos e checamos se
     a equivalência se permanece. Vimos isso na sequência.


     Supondo que para essa EDO podemos usar o fator integrante \( \mu = \mu(x, y) \), então
     \( \widetilde{M} + \widetilde{N} y'   \mu M + \mu N y' = 0 \) 
     \[ \mu(x, y) \left[ 2 \sin (x) - x \right]y' + \mu(x, y) (-\tan(y)) = 0 \]
     
     Checando a validade de \( \frac{\partial }{\partial y} \widetilde{M}  = \frac{\partial }{\partial x} \widetilde{N}\):
     
     
     \[ - \frac{\partial \mu}{\partial y} \tan(y) - \mu \sec^2 (y) = \frac{\partial \mu}{\partial x} \left( 2 \sin (y) - x \right) - \mu (-1) \]
     \[ \Leftrightarrow \left( x - 2 \sin(y) \right) \frac{\partial \mu}{\partial x} - \tan(y) \frac{\partial \mu}{\partial x} = (\sec^2 (y) - 1) \mu  \]
     
     \paragraph{$\rightarrow $ Ansatz}: \( \mu = \mu(x)\Rightarrow \frac{\partial \mu}{\partial x} = \mu' \)  e \( \frac{\partial \mu}{\partial y} = 0 \) 
     
     \[ \Leftrightarrow (x - 2\sin(y)) \mu'(x) = \left( \tan(y) \right)^2 \mu(x) \]
     \[ \Leftrightarrow \frac{\mu'(x)}{\mu(x)} = \frac{ \left( \tan(y) \right)^2 }{x - 2 \sin(y)} \]

     Essa última linha é uma contradição, já que o lado direito não depende explicitamente apenas de
     \( x \), ou seja, esse chute não é válido.
     

     \paragraph{$\rightarrow $ Ansatz}: \( \mu = \mu(y) \Rightarrow \frac{\partial \mu}{\partial x}  = 0, \frac{\partial \mu}{\partial y} = \mu'(y)\) 

     \[ - \tan(y) \mu' = \left( \tan(y) \right)^2 \mu \Rightarrow \frac{\mu'(y)}{\mu(y)}= - \tan(y) \Rightarrow \mu(y) = \cos(y)\]



     \paragraph{$\blacksquare$ Exemplo}: $(2 \sin(y) \cos(y) - x \cos (y))y' - \sin(y) = 0$

     \[ M_y = - \cos (y) , N_{x} = - \cos(y) \]

     \[ M = \psi_{x} = - \sin (y) \Rightarrow \psi (x, y) = - x \sin (y) + h(y)\]
     \[ N = \psi_{y} = \sin(y) - x \cos(y) = - x \cos(y) + h'(y) \Rightarrow h'(y) = \sin(2y) \Rightarrow h(y) = -
       \frac{1}{2} \cos(2y) \]
     \[  \psi(x, y) = - x \sin(y) - \frac{1}{2} \cos(2y) + \mathbb{C} \]

     Logo, a solução implicita da EDO é

     \[ -x \sin(y(x)) - \frac{1}{2} \cos(2 y(x)) = \mathbb{C}  \]

     
%%% Local Variables:
%%% mode: latex
%%% coding: utf-8
%%% TeX-master: "main.tex"
%%% End:


