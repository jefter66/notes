\section{Mini revisão de séries de Taylor}
\subsection{Multiplicação de duas séries}
  Dadas as funões analiticas em \( x_0 \) , $f(x)$ e $g(x)$, com
  \[ f(x) = \sum_{n = 0}^{\infty} a_n (x - x_0)^n \text{ e } g(x) = \sum_{n = 0}^{\infty}b_n (x - x_0)^n \] 
  vale a seguinte relação
  \[ f(x) g(x) := \bigg[\sum_{n = 0}^{\infty} a_n (x - x_0)^n  \bigg] \bigg[\sum_{n = 0}^{\infty} b_n (x - x_0)^n  \bigg] \]

  \[ f(x) g(x) := \sum_{n = 0}^{\infty} c_n (x-x_0)^n \]
  onde \( c_n = a_0 b_n + a_1 b_{n-1} + \dots + a_n b_0 \) .

\subsection{Série de taylor em torno de um ponto \( x_0 \):}
\[ \boxed{ f(x) =  \sum_{n = 0}^{\infty} \frac{f^{(n)}(x_0)}{n!} (x-x_0)^n } \]
\section{Solução em torno de pontos regulares}
\subsection{Introdução e analiticidade de séries de potências}
 Dada a série de potências $ \sum a_n (x-x_0)^n$, se a série
 converge para valores de \( x \) num intervalo \( x_0 - \delta, x_0 + \delta \) , \( \delta > 0 \) ,
 então a série é analitica em \( x = x_0 \) .
 \paragraph{$\blacksquare$ Exemplo}: Dada a EDO \( a(x) y''(x) b(x)y'(x) + c(x)y(x) = 0 \)
 \[ y''(x) + p(x) y'(x) + q(x) y(x) = 0  \]
 

 \begin{enumerate}
  \item[] Escolher o ponto \( x_0 \) que vamos analisar.
  \item[] Estudar a analiticidade da série.
  \end{enumerate}

\paragraph{$\blacksquare$ Exemplo}:  \( (1 - x) y'' + y = 0  \)

  \[ y'' + \frac{1}{1 - x}y = 0, p(x) = \frac{1}{1 - x} = \sum_{n = 0}^{\infty} x^n \]

  Pelo critério da razão: \( \lim_{n \infty} |r_n| = |x| \), convergente no intervalo \( (-1, 1) \)
  e convergente no ponto \( 0 \) . Escrevendo a EDO em formato de séries:

  \[ \sum_{n = 0}^{\infty} (n + 2)(n+ 1) a_{n+2} x^n + \bigg[ \sum_{n = 0}^{\infty} x^n \bigg]
  \bigg[ \sum_{n = 0}^{\infty} a_n x^n \bigg] = 0\]

\subsection{Solução em séries em torno de pontos regulares}

 \paragraph{$\blacksquare$ Exemplo}: \( y''(x) + \sin(x) y(x) = 0 \) 

    Temos \( p(x) = 0 \) e \( q(x) = \sin(x) = \sum_{n = 0}^{\infty} (-1)^n \frac{x^{(2n +
    1)}}{(2n + 1)!} \), ambas funções são analiticas em torno de \( x_0
    \). Escolhendo
    \( x_0 = 0 \) temos:

    \[ \sum_{n = 0}^{\infty} (n + 2)(n+1) a_{n + 2} x^n + \left[ \sum_{n = 0}^{\infty} (-1)^n
    \frac{x^{(2n + 1)}}{(2n + 1)!} \right] \left[ \sum_{n = 0}^{\infty} a_n x^n \right] = 0\]

    Nesse tipo de resolução podemos definir uma ordem n e tentar encontrar o termo geral \( a_n \) e
    então retornar à série original de \( y(x) \) nas condições iniciais dadas.
    

    Abrindo os somatórios e agrupando os termos a partir das menores potências de \( x \) até \( x^3 \)  ;

    \[ (2 \cdot 1 a_2 + 3 \cdot 2 a_3 x + 4 \cdot 3 a_4 x^2 + 5 \cdot  4 x^3 + \cdots) +
    \left[ x - \frac{x^3}{3!} + \frac{x^5}{5!} - \cdots  \right] \left[ a_0 + a_1 x + a_2 x^2 + a_3
    x^3 + \cdots  \right]\]
    \[ 2 a_2 + (6 a_3 + a_0) x + (12 a_4 + a_1) x^2 + (20 a_5 + a_2 - \frac{a_0}{6})x^3 +
    \mathcal{O}(x^4) = 0 \]

    Relação de independência linear, ou seja, os coeficientes *devem* ser nulos.

    \[ \begin{cases}
    a_2 = 0 \\
    6a_3 + a_0 = 0 \\
    12 a_4 + a_1 = 0 \\
    \vdots 
    \end{cases}  \Rightarrow  \begin{align} a_3 = - \frac{a_0}{3} \\ a_4 = - \frac{a_1}{12} \\ a_5
    =  - \frac{a_0}{120}\end{align}\]
  
    Escrevendo \( y(x) \) em termo dos coeficientes encontrados: 

    \[ y(x) =  a_0 + a_1 x  - \frac{a_0}{6}x^3 - \frac{a_1}{12}x^4 + \frac{a_0}{120} x^5 + \cdots \]
    \[ y(x) = a_0 \left[ 1 - \frac{x^3}{3!} + \frac{x^5}{5!} + \cdots  \right] + a_1 \left[ x -
    \frac{x^4}{12} + \cdots  \right]\]

    Onde \( a_0 \) e \( a_1 \) são as condições de contorno.

    \paragraph{$\blacksquare$Exemplo}: \( y'' + k^2 y = 0, k \in \mathbb{R} \text{ e } y (x) = ?  \)
    Exemplo de uma EDO que podemos resolver sem o método de séries. Caso do movimento oscilatório.


    De forma direta, temos

    \[ \sum_{n = 0}^{\infty} \left[ (n+2) (n+1) a_{n + 2} x^n \right]  + k^2 \sum_{n = 0}^{\infty} a_n x^n = 0\]
    \[ \sum_{n = 0}^{\infty} \left[ (n+2)(n + 1) a_{n + 2} + k^2 a_n \right] x^n = 0  \]
    \[ a_{n + 2} = - k^2 \frac{a_n}{(n + 2)(n + 1)}, n = 0, 1, 2, ... \]

     $$ \begin{flalign*}
    a_2 & = - k^2 \frac{a_0}{2} &\\
    a_3 & = -k^2 \frac{a_1}{3 \cdot  2} &\\
    a_4 & = -k^2 \frac{a_2}{4 \cdot 3 } = k^4 \frac{a_0}{4 \cdot  3 \cdot  2 \cdot  1 }&\\
    a_5 & = -k^2 \frac{a_3}{5 \cdot 4 } = k^4 \frac{a_1}{5 \cdot  4 \cdot  3 \cdot  2 }
    \end{flalign*} $$

    Separando em termos pares e ímpares:

    \item{$\blacksquare$} n par: \( n = 2j, j = 0, 1, 2, ... \)
      \[ a_{2j + 2} = - k^2 \frac{a_{2j}}{(2j + 2)(2j + 1)} = - \frac{k^2}{(2j+2)(2j+1)} \left[ -
      \frac{k^2 a_{2j - 2}}{(2j)(2j -1)} \right]  \]
      \[ a_{2j + 2} = \frac{(-k^2)(-k^2)\cdot  \cdots (-k^2) a_0}{(2j+2)(2j + 1)(2j)(2j - 1) \cdots
      2)} = \frac{(-k^2)^{j + 1} a_0}{(2j + 2)!}\]
      \[ a_{2j + 2} = \frac{(-k^2)^{j + 1} a_0}{[2(j + 1)]!} \Rightarrow \boxed{ a_{2j} = (-1)^j\frac{
      k^{2j} a_0}{(2j)!}} \]
      
    \item{$\blacksquare$} n ímpar: \( n = 2j + 1 \) 

      \[ a_{2j + 3} = - \frac{k^2 a_{2j + 1}}{(2j + 3)(2j + 2)} = \frac{(-k^2)(-k^2) a_{2j-1}}{(2j+3)(2j+2)(2j+1)(2j)\cdots } \]
      \[ a_{2j + 3} =  \frac{(-k^2)(-k^2) \cdots (-k^2) a_1}{(2j+3)(2j+2)(2j+1)\cdots  3 \cdot  2} \]
      \[ a_{2j+3} = \frac{(-1)^j (k^2)^{j + 1} a_1}{(2j + 3)!} \Rightarrow \boxed{ a_{2j + 1} =
      (-1)^{j} \frac{(k)^{2j} a_1}{(2j + 1)!}} \]

    Voltando à série original, temos
    \[ y(x) = \sum_{n = 0}^{\infty} a_n x^n = \sum_{\text{n par}} a_n x^n + \sum_{\text{n impar}}
    a_n x^n \]
    
    \[ y(x) = a_0 \sum_{n = 0}^{\infty} (-1)^n \frac{k^{2n} x^{2n}}{(2n)!} + \underbrace{a_1 \sum_{n =
    0}^{\infty}(-1)^n \frac{k^{2n} x^{2n + 1}}{(2n + 1)!}}_{= \frac{a_1}{k} \left[ kx - \frac{k^3
    x^3}{3!} + \frac{k^5 x^5}{5!} - \frac{k^7 x^7}{7!} + \cdots  \right]} \]

    \[ y(x) = a_0 \cos(kx)  + \frac{a_1}{k} \sin(kx) \]

    \[ \boxed{ y(x) = A \cos(kx) + B \sin (kx) } \]

\section{Método de séries para pontos singulares regulares}



\[ y'' + p(x) y' + q(x) = 0 \]

Se \( p(x) \) e \( q(x) \) forem analíticas em \( x = x_0 \), então \( x = x_0 \) é um \textbf{ponto
ordinário} e vale a solução do tipo
\[ y(x) = \sum_{n = 0}^{\infty} a_n (x - x_0)^n \]
caso contrário, \( x = x_0 \) é um ponto singular, portanto precisamos de um outro tratamento
para poder escrever a solução no formato de série de potências.

Se \( p(x) \) \textbf{ou} \( q(x) \) não forem analíticas em \( x = x_0 \) , mas \( (x - x_0) p(x) \) e \(
(x-x_0)^2 q(x) \) forem, então
\( x = x_0 \) é dito ser um *ponto singular regular*. Caso contrário \( x = x_0 \) é um ponto
singular irregular \emph{(esse caso não é tratado nessa disciplina, então não tentei buscar entender ainda...)}.

  \paragraph{$\blacksquare$ Exemplo}: \( x^2 y'' + 2x y' + 3y = 0 \Leftrightarrow y'' + \frac{2}{x}y ' + \frac{3}{x^2}y = 0 \)

  \[ xp(x) = 2 \text{ e } x^2 q(x) = 3 \]

  No caso \( x = x_0 \) ser ponto singular regular, procuramos uma solução em série de
  potências com a seguinte forma:


  \begin{equation}
    y(x) = (x - x_0)^{\lambda} \sum_{n = 0}^{\infty} a_n (x - x_0)^n, a_0 \neq 0
    \label{eq:solucao_singular_regular}
  \end{equation}


  Impomos que ao menos uma das soluções devem ser dessa forma e caso o \( \lambda = 0  \), ou seja,
  conseguimos apenas uma solução a partir dele, podemos chegar à outras soluções por outros métodos
  como redução de ordem ou Wronksiano...

  Expandindo \( (x - x_0) p(x) \) e \( (x - x_0)^2 q(x) \) em séries de potências:


  $$ \begin{cases}
       (x - x_0) p(x) = \sum_{n = 0}^{\infty} p_n (x - x_0)^n \\
       (x - x_0)^2 q(x) = \sum_{n = 0}^{\infty} q_n (x - x_0)^n
     \end{cases} $$
     
     
     \[ y(x) = \sum_{n = 0}^{\infty} a_n (x - x_0)^{n + \lambda}\]
     \[ y'(x) = \sum_{n = 0}^{\infty} (n + \lambda)a_n (x-x_0)^{n + \lambda - 1} \]
     \[ y''(x) = \sum_{n = 0}^{\infty} (n+\lambda)(n + \lambda - 1) a_n (x - x_0)^{n + \lambda - 2} \]
     
     Saindo da forma canônica multiplicando a EDO por (x - x_0)^2:

     \[ (x - x_0)^2 y'' + (x-x_0) \left[ (x-x_0) p(x) \right]y' + \left[ (x-x_0)^2 q(x) \right]y = 0  \]
     
     \begin{multline*}
       (x-x_0)^2 \sum_{n = 0}^{\infty} (n+\lambda)(n + \lambda - 1) a_n x^{n + \lambda - 2} + (x-x_0)
       \sum_{n = 0}^{\infty} \left[ (x - x_0) p(x) \right] \left( (n + \lambda)a_n(x-x_0)^{n + \lambda - 1}
       \right) \\ + \sum_{n = 0}^{\infty} \left[ (x-x_0)^2 q(x) \right] a_n (x - x_0)^{n + \lambda} =  0
     \end{multline*}
     
     Fazendo a expansão em série de Taylor para \( (x - x_0) p(x)  \) e \( (x - x_0)^2 q(x) \) de forma a
     considerar apenas o termo mais baixo \( p_0, q_0 \), temos que
     \[ (x - x_0)p(x) = p_0 + (x-x_0) p_1 + (x - x_0)^2 p_2 \cdots \Rightarrow p_0 = \lim_{x\to x_0}
       (x-x_0) p(x)  \]
     \[ q_0 = \lim_{x\to x_0} (x-x_0)^2 q(x) \]
     
     
     Rearranjando as séries acima e simplificando as potências temos a seguinte série:
     
     $$ \sum_{n = 0}^{\infty} \left[ (n + \lambda) (n + \lambda - 1)a_n + [(x - x_0) p(x)] (n +
       \lambda)a_n + [(x-x_0)^2 q(x)] a_n \right] \left(x - x_0 \right)^{n + \lambda}=0 $$

     $$
     \sum_{n = 0}^{\infty} \left[ (n + \lambda)(n + \lambda - 1) + \left[ (x - x_0)p(x) \right] (n +
       \lambda) + \left[ (x-x_0)^2 q(x) \right] \right] a_n (x - x_0)^{n + \lambda} = 0
     $$


     \[  \sum_{n = 0}^{\infty} \left[ (n + \lambda )(n + \lambda - 1) + (n + \lambda ) p_0 + q_0 \right]
       a_n (x - x_0)^{n + \lambda} \]
     
     Abrimos as séries e escrevendo as combinações em termos de ordem mais baixa. Isso é útil porque
     assim
     podemos fatorar e chagar numa relação para \( \lambda  \):
     
     \[ \lambda (\lambda -1) a_0 \left( x - x_0 \right)^{\lambda } + p_0 \lambda a_0 (x -
       x_0)^{\lambda} + q_0 a_0 (x - x_0)^{\lambda } + \mathcal{O} (x - x_0)^{\lambda + 1} =  \]
     \[ a_0 \left[ \lambda (\lambda - 1) + \lambda p_0 + q_0  \right] \left( x - x_0 \right)^{\lambda } +
       \mathcal{O}(x - x_0)}^{\lambda + 1} = 0\]
   
   Podemos escrever a relação: 
   
   \begin{equation}
     \lambda \left( \lambda - 1 \right) + \lambda p_0 + q_0 = 0
     \label{eq:equacao_indicial}
   \end{equation}
   
   
   \paragraph{$\blacksquare$ Exemplo} : \( y'' + \frac{2}{x}y' + \frac{3}{x^2}y = 0, x_0 = 0 \) 
   
     \[ p(x) = \frac{2}{x} \Rightarrow xp(x) = 2 = p_0 + p_1 x + p_2 x^2 + ... \]
     
     \[ q(x) = \frac{3}{x^2} \Rightarrow x^2 q(x) = 3 = q_0 + q_1 x + q_2 x^2 + ... \]
     
     Logo, \( x_0 = 0 \) é um ponto singular regular e \( y(x) = x^\lambda \sum_{n = 0}^{\infty} a_n x^{n} \) .
     
     Encontrando o coeficiente \( \lambda  \) usando a equação indicial (\ref{eq:equacao_indicial}) :
     \[ \lambda (\lambda - 1) + p_0 \lambda  + q_0 = 0 \Rightarrow \lambda ^2 + \lambda  + 3 = 0 \]
     \[ \lambda  = \frac{-1 \pm \sqrt{-11}}{2}  \Rightarrow \lambda _1 = - \frac{1}{2} \pm
       \frac{i\sqrt{11}}{2}, \lambda _2 = - \frac{1}{2} - \frac{i\sqrt{11}}{2}\]
     
     \[ y(x) = y_1(x) + y_2(x) = x^{\lambda _1} \sum_{n = 0} a_n x^n + x^{\lambda _2} \sum_{n =
         0}^{\infty} a_n x^n \]
     \subsection{Casos para \( \lambda  \)}
     Esse indice é definido a partir da relação (\ref{eq:equacao_indicial}) e para cada solução da
     equação quadrática tem significados diferentes. Eles foram divididos em alguns casos.

     Nessa etapa do curso muitos desses resultados não foram rigorosamente justificados e
     em alguns deles não fui atrás de tentar entender como chegar nessas relações por
     conta própria. :p 😛
     
     \subsubsection{Caso \( \lambda  \pm \notin \mathbb{R}   ( \lambda _{-} = \lambda _{+}^{*}) \)}
     A solução de \( \lambda  \) é um complexo. Nesse caso o complexo conjugado também é válido, mas
     como queremos apenas soluções reais \( y(x) \) , precisamos tratar o caso complexo
     a fim de conseguir apenas a perte real da solução.
     Para isso utilizamos as seguintes soluções:
     
     \[ y_1 (x) = | x - x_0 |^{\lambda _+} \sum_{n = 0}^{\infty} a_n (x - x_0)^n, a_0 \neq 0 \]
     \[ y_2 (x) = | x - x_0|^{\lambda _{-}} \sum_{n = 0}^{\infty} b_n (x - x_0)^{n}, b_0 \neq 0 \]
     
     
     \subsubsection{Caso \( \lambda  \pm \in \mathbb{R} ( \lambda _+ \geq \lambda _-) \) }
     
     
     \begin{enumerate}
     \item[$\blacksquare$] Subcaso: \( y_1 (x) = |x - x_0 |^{\lambda _+} \sum_{n = 0}^{\infty} a_n (x - x_0)^n \)
       
       Nesse caso a segunda solução pode ser construida a partir da primeira e
       fica em termos de logaritmo e outra série. Só utilizamos o resultado.
       Se em algum momento eu estudar as demonstrações posso coloca-las aqui.

       \[ y_1 (x) = y_1(x) \ln |x - x_0| + |x - x_0|^{\lambda _+} \sum_{n = 0}^{\infty} b_n (x - x_0)^n \]
       
       Além disso, para definir os coeficientes \( b_n \) é necessário substitutir a solução na EDO
       original e encontrar o termo geral. Ao menos foi isso que entendi.
     
     
     \item[$\blacksquare$] Subcaso:\( \lambda _+ - \lambda _{-} \in \mathbb{N}^{*} \)       
       
       \[ y_1(x) = |x - x_0|^{\lambda } \sum_{n = 0}^{\infty} a_n (x - x_0)^n \]
       \[ y_2 (x) =  \alpha y_1(x) \ln |x - x_0| + |x - x_0|^{\lambda} \sum_{n = 0}^{\infty} b_n (x - x_0)^n \]
       
 
     \item[$\blacksquare$] Subcaso:  \( 0 < \lambda _{+} - \lambda _{-} \notin \mathbb{N} \)
       
       
       \[ y_1(x) =  | x - x_0|^{\lambda _+} \sum_{n=  0}^{\infty} a_n (x - x_0)^n\]
       \[  y_2 ( x) =  |x - x_0|^{\lambda_ - } \sum_{n = 0}^{\infty} b_n (x - x_0)}^n \]
     
   \end{enumerate}

   
   \paragraph{$\blacksquare$Exemplo}:   Equação de Bessel
   \[ x^2 y'' + x y' + (x^2 +  \nu^2) y = 0; \nu \in \mathbb{R}_{+} \]
   
   Para \( x_0 = 0  \) 
 
   \[  y'' + \frac{1}{x} y' + \left( 1 - \frac{\nu^2}{x^2} \right) y = 0 \]
   \[ p(x) = \frac{1}{x} \]e \( q(x) = 1 - \frac{\nu^2}{x 2} \) 
   
   Não são analiticas em \( x_0 \) , encontrando os pontos singulares regulares:

   \[ x p(x) = 1 \text{  e } x^2 q(x) = x^2 - \nu^2 \]
   
   essas funções são analiticas em \( x_0 \) .
   Resolvendo a equação indicial:
   
   \[ \lambda (\lambda - 1) p_0 \lambda + q_0 = 0 \]
   \[ p_0 =  \lim_{x\to x_0} xp(x) = 1 \text{ e }  q_0 = \lim_{x \to x_0} x^2q(x) = - \nu^2\]
   
   \[ \lambda (\lambda - 1) + \lambda - \nu^2 = 0  \]
   \[ \lambda ^2 - \nu^2 = 0 \Rightarrow \lambda  = \pm \nu  \]
   (Se o \( \mu \) não tivesse sido definido como estritamente positivo a solução seria \( \lambda  =
   \pm | \nu|\)).
   

   \paragraph{$\blacksquare$ Exercicio proposto}:  resolver para \( \nu = 0, \nu = \frac{1}{2}, \nu = 1 \) 
   
 
%%% Local Variables:
%%% mode: latex
%%% coding: utf-8
%%% TeX-master: "main.tex"
%%% End:


  