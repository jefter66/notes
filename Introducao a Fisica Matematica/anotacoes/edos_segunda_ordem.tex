\section{Lineares homogêneas}

Os métodos para resolver as EDOs de segunda ordem lineares são os métodos
do Wronskiano (usando o teorema de Abel), redução de ordem, variação de parametros (para
as EDOs inomogêneas) e em último caso, método de séries. Nessa etapa vimos todos métodos exceto esse
último, que será visto na sequência.

As EDOs de segunda ordem tem a seguinte cara:
\[ a(t) y'' + b(t) y' + c(t) y = 0  \]

vamos considerar inicialmente só o caso homogêneo por simplicidade e sempre
adotar o uso da equação na forma ``canônica'', isto é, com a derivada de maior ordem multiplicada
por \( 1 \):

\[ y''(t) + p(t) y'(t) + q(t) y(t) = 0, p(t) = \frac{b(t)}{a(t)}, q(t) = \frac{c(t)}{a(t)} \]


As EDOs de segunda ordem são compostas de 2 soluções linearmente independentes, isso significa que
podemos construir uma solução geral a partir de outra e em geral é isso que vamos fazer, supor uma
possível solução, com base por exemplo na ``cara'' da EDO.Isto vai ser visto melhor na prática em
exemplos desenvolvidos.

Segue propriedades gerais das EDOs de segunda ordem lineares:


\begin{theorem}[Linearidade ou principio da superposição]
  Se \( y_1 (t)  \)  e \( y_2 (t) \) são soluções de uma EDO linear homogênea, então
  \( \alpha y_1 (t) + \beta y_2 (t) \), \( \forall \alpha , \beta  \in \mathbb{R} \), também é solução.
\end{theorem}

\begin{theorem}[Solução geral]  
  Se \( y_1(t) \) e \( y_2 (t) \) são soluções de uma EDO linear homogênea de 2ª ordem que não
  são proporcionais uma à outra, então \( y(t) = \alpha y_1 (t) + \beta y_2 (t) \), com \( \alpha  \) e
  \( \beta  \) constantes quaisquer é a solução geral da EDO dada.
\end{theorem}


\begin{theorem}[Wronskiano]\label{teo:definicao_wronskiano}
  Dadas duas funções quaisquer \( y_1(t) \) e \( y_2(t) \) define-se o chamado
  Wronskiano como sendo a função
  \begin{equation}
    W[y_1, y_2] (t) := y_1 (t) y_2 '(t) - y_2 (t) y_1 ' (t)
    \label{eq:wronskiano_formula}
  \end{equation}
\end{theorem}

\paragraph{Expandindo um pouco a ideia do Wronskiano}

Nota-se que, \( W = 0 \) só e só se \( y_1 (t) = f(t) \) e \( y_2 (t) = c f(t)\) .

Derivando o Wronskiano:
\[ \frac{d W }{d t} = \frac{d  }{d t} \left[ y_1 y_2 ' - y_2 y_1 ' \right] = y_1'y_2 ' + y_1 y_2''
  - y_2' y_1' - y_2 y_1'' \]
\[ \frac{d W }{d t} = y_1 y_2'' - y_2 y_1'' \]

\[ \frac{d W }{d t} = y_1 \left\bigg( -p(t) y_2 ' - q(t) y_2  \right) - y_2 \left\bigg( -p(t) y_1' - q(t) y_1 \right)\]

\[ \frac{d W }{d t} = -p(t) \left\bigg( y_1 y_2 ' - y_2 y_1' \right) = -p(t) W\]

\begin{theorem}[Teorema de Abel]\label{eq:teorema_abel}
  Se \( y_1 \) e \( y_2 \) são soluções da EDO \( y'' + p(t) y' + q(t) y = 0 \) seu Wronskiano
  satisfaz \( W' + p(t) W = 0 \) vale
  \begin{equation}
    W[y_1, y_2] (t) = W_0 \exp\left({- \int p(t) dt}\right)
    \label{eq:teorema_abel}
  \end{equation}
\end{theorem}

\paragraph{$\blacksquare$ Exemplo}: \( x^2 y'' - x(x + 2)y' + (x+2) y = 0, x > 0, y(x) = \)?
Sabemos que a solução geral deve ser da forma \( y(x) = \alpha y_1 (x) + \beta y_2 (x) \) 
\[ y'' - \frac{x(x+2)}{x^2}y' + \frac{(x + 2)}{x^2}y = 0 \]
\[ y'' - \left( 1 + \frac{2}{x} \right)y' + \frac{1}{x} \left( 1 - \frac{2}{x} \right) y = 0 \]

\[ W = y_1 y_2' - y_2 y_1' = W_0 \exp{\int \left( 1 + \frac{2}{x} \right) dx} = W_0 \exp{x + 2 \ln(x)} \]
\[ y_1 y_2' - y_2 y_1' = W_0 x^2 e^x \]

Chutando uma solução para a EDO:
Olhando para a essa EDO podemos chutar uma solução do tipo \( y(x) = x^{\lambda} \), já que derivando duas
vezes vamos fazer o coeficiente \( x^2 \) sumir da EDO. Primeiro fazemos essa substituição na EDO
original e tentar calcular o \( \lambda  \):

\[ \begin{cases*}
     y' = \lambda x^{\lambda - 1} \\
     y'' = \lambda (\lambda -1) x^{\lambda - 2} 
   \end{cases*} \Longrightarrow 
   \begin{align*}
     \lambda (\lambda - 1) x^{\lambda } - \lambda (\lambda + 2) x^{\lambda } + (x+2)x^{\lambda } & = 0  \\
     x^{\lambda} \left[ \lambda (\lambda -1) - 2 \lambda  + 2 \right] + (1 - \lambda ) x^{\lambda + 1} &= 0
   \end{align*}
 \]

 Se \( \lambda  = 1 \) a relação é válida, logo \( y_1(x) = x \) pode ser uma solução.


 A partir desse resultado usamos o (\ref{eq:teorema_abel}) para encontrar a segunda solução. Caimos
 em uma EDO de primeira ordem e resolvendo ela temos \( y_2(x) \), e portanto a solução completa.

 \[ y_1 y_2' - y_2 y_1' = W_0 x^2 e^x \]
 \[ x y_2'- y_2 = W_0 x^2 e^x \Longrightarrow y_2' - \frac{1}{x} W_0 x e^x, W_0 \neq 0\]

 \[ \mu = \frac{1}{x} \Longrightarrow  \frac{d  }{d x} \left( \frac{y_2}{x} \right) = e^x \Longrightarrow  \frac{y_2}{x} = e^x + \mathbb{C} \]
 \[ \Longrightarrow y_2(x) = x e^x \]

 Logo, a solução geral da EDO dada é \( y(x) = \alpha y_1(x) + \beta y_2(x) \) 
 \[ y(x) = \alpha x + \beta x e^x \]

 A partir das condições iniciais podemos definir os coeficientes \( \alpha \) e \( \beta  \).

\subsection{Método de redução de ordem}


Partindo de uma EDO de segunda ordem conseguimos encontrar uma solução a partir de outra
substituindo por uma EDO de primeira ordem.

\[ a y'' + b y' + cy = 0 \]

Supondo que conhecemos uma das soluções \( y_1 \) que obtemos a partir de algum Anzats. Podemos
então obter a \( y_2 \) por redução de ordem fazendo \( y_2 = v(t)y_1(t) \) e substituindo na EDO
original
\[ \begin{cases}
     y_2 ' = v' y_1 + v y_1' \\
     y_2 '' = v'' y_1 + 2v'y_1' + v y_1''
   \end{cases}
   \Longrightarrow
   \begin{align*}
     a \left\bigg( v''y_1+2v'y_1'+vy_1'' \right) + b \left\bigg( v'y_1+vy_1' \right) + c v y_1 & =
                                                                                                 0\\
     v \underbrace{\left[ ay_1'' + by_1' + cy_1 \right]}_{= 0} + av''y_1+2v'y_1'+bv'y_1 & = 0 \\
   \end{align*}
 \]

 \[
 av''y_1 + 2v'y_1' + bv' y_1 = 0 \Longrightarrow  v'' + v' \left\bigg[ \frac{2y_1' + b y_1}{ay_1} \right] = 0
 \]

 a partir desse ponto substituimos o \( y_1, a \) e \( b \), que são conhecidos e resolvemos a EDO
 em \( v'' \) para depois substituir em \( y_2 \). Não continuei essa demonstração porque me parece
 ser mais útil tratar um exemplo real ao invés.


 \paragraph{$\blacksquare$ Exemplo}: \( 4 y'' + 4y' + y = 0 \)
 
 Supondo \( y_1 = e^{r t} \Rightarrow y_1' = ry_1 \Rightarrow y_1'' = r^2 y_1\) e disso segue que \( y \left( 4r^2 + 4r
   + 1 \right) = 0 \Rightarrow (2r + 1)^2 = 0 \Rightarrow r = -\frac{1}{2} \) é uma solução. Aplicamos a redução de
 ordem para encontrar a segunda solução.

 \[ y_2 = v y_1 \Rightarrow \begin{cases}
                    y_2' = v' y_1 + vy_1' \\
                    y_2'' = v''y_1 + 2v'y_1' + vy_1''
                  \end{cases}  \]

       
  \[ \left( v'' y_1 + 2 v' y_1' + vy_1'' \right)  + p  () + q \left( vy_1 \right)  =  0\]                
  \[ \Rightarrow v'' y_1 + v' \left( 2y_1' + p y_1 \right) = 0  \]

  Resolvendo essa EDO para encontrar \( y_2 \) :
  \[ y_1' = - \frac{1}{2} y1 , y_1'' = \frac{1}{4} y_1 \]
  \[ y_2 = vy_1 \Rightarrow \begin{cases}
                    y_2' = v' y_1 - \frac{v}{2}y_1'\\
                    y_2'' = v''y_1 - \underbrace{\frac{v'y_1}{2}- \frac{vy_1'}{2}}_{= - v y_1} + \frac{vy_1}{4}
                  \end{cases} \]

  Substituindo na equação original:
                
\[ 4 \left\right( v'' y_1 - v y_1 + \frac{v}{4}y_1 \right) + 4 \left\bigg( v' y_1 - \frac{v}{2}y_1
\right) + v y_1 = 0 \] 
\[ \Rightarrow 4 v'' y_1 = 0 \Rightarrow v'' = 0 \Rightarrow v' = a \Rightarrow v = a t + b \]

\[ y_2 = v y_1 = at e^{-t/2} + b e^{-t/2} \]
                
\section{Lineares e inomogêneas}
\subsection{Solução com coeficientes constantes}
Para EDOS com a cara
\[ a y''+ b y' + c y = 0 \]
onde \( a, b \)  e \( c \) são constantes, soluções podem sembre ser dadas
pelo chute \( y_1(x) = e^{r x} \) e a solução geral \( y(x) = C_1 e^{r_1 t} + C_2 e^{r_2 t}  \).

Substituindo na EDO genérica dada:
\[ a r^2 e^{rx} + b r e^{rx} + c e^{rx} = 0  \]
\[ e^{rx} \left( a r^2 + b r + c \right) = 0 \]

\[ \Leftrightarrow ar^2 + br + c = 0 \]


\begin{theorem}[Equação caractéristica de uma EDO]
  Dada uma EDO linear homogênea de segunda ordem com coeficientes constantes:
  $a y'' + b y' + cy = 0$, a equação \( a r^2 + br + c = 0 \) é a equação caractéristica
  da EDO dada.
\end{theorem}

Sejam \( r_1  \) e \( r_2 \) solução da equação caractéristica, temos três possíveis casos:

\paragraph{$\blacksquare$ Caso} \( r_1, r_2 \in \mathbb{R} \)  e \( r_1 \neq r_2 \): \( y(x) =  \alpha e^{r_1 x} + \beta e^{r_2 x}\) 
\paragraph{$\blacksquare$ Caso} \( r_1 = r_2 \), usar método do Wronskiano ou redução de ordem.
\paragraph{$\blacksquare$ Caso} \( r_1 \neq r_2 \)  e \( r_1 = \overline{r_2} \in \mathbb{C} \): \( y(x) = \alpha e^{r_1x} + \beta e^{r_2 x} \) 
  Nesse caso é necessário fazer uma mudança de coordenadas e passar a trabalhar com coordenadas
  polares e tratar usando formúla de Euler. Lembrando que as EDOs que estamos trabalhando estão em
  \( \mathbb{R} \), logo temos que sempre tomar cuidado em checar se a solução final está nesse domínio.

\subsection{Solução para coeficientes não constantes}
Equações com a cara 
\[ a(t) y''(t) + b(t) y'(t) + c(t) y(t) = f(t) \]

Solução geral: \( y(t) = y_{H} (t) + y_{P} (t) \) 

onde \( y_H \) é uma solução geral da EDO homogênea associada (obtida substituindo
$f(t)$ por zero e resolvendo a EDO homogênea resultante) e \( y_P \) é uma solução particular
qualquer da EDO inomogênea dada.

Supondo que já resolvemos a EDO homogênea por algum método:

\[ y_H(t) = \alpha  y_1 (t) + \beta y_2(t) \]
onde já conhecemos as funções \( y_1 \) e \( y_2 \) .


\subsection{Método da variação de parâmetros para encontrar \( y_P(t) \)}

Reescrevendo a solução particular da inomogênea como :

\[ y_P(t) = \alpha (t) y_1(t) + \beta (t) y_2(t) \]

E colocando a EDO original na forma canônica:
\[ y_P' = \alpha y_1 ' + \alpha ' y_1 + \beta  y_2 ' + \beta ' y_2 \]

Impomos a relação:
\begin{equation}
  \alpha ' y_1 + \beta ' y_2 = 0  
  \label{eq:inomogeneas_1}
\end{equation}

segue que \( y_P ' = \alpha (t) y_1 '(t) + \beta (t) y_2 ' (t) \) 


\[ y_P '' =  \alpha ' y_1 ' + \alpha y_1 ''  + \beta ' y_2 ' + \beta y_2 '' \]


Substituindo na EDO dada:
   
\[ y_P '' + p y_P ' + q y_P  = \alpha y_1 '' + \alpha ' y_1 ' + \beta y_2 '' + \beta ' y_2 ' + p
  \left( \alpha y_1 ' + \beta  y_2 '  \right) + q \left( \alpha y_1 + \beta  y_2 \right) = r(t) \]
\[ \alpha (t) \underbrace{\left[ y_1 '' (t) + p(t) y_1' (t)  + q(t) y_1(t) \right]}_{ =  0}  + \beta (t) \underbrace{\left[
      y_2''(t) + p(t) y_2 ' (t) + q(t) y_2 (t)  \right]}_{ = 0 } + \alpha ' y_1' + \beta ' y_2' = r(t) \]

\begin{equation}
  \alpha ' y_1 ' + \beta ' y_2 ' = r(t) 
  \label{eq:inomogeneas_2}
\end{equation}


Com as duas equações (\ref{eq:inomogeneas_1}) e (\ref{eq:inomogeneas_2}) temos o sistema linear:

\[ \begin{cases}
     y_1 \alpha ' + y_2 \beta ' = 0 \\
     y_1 ' \alpha ' + y_2 ' \beta ' = r(t)   
   \end{cases}\]
 
 \[ \begin{bmatrix}
      y_1 & y_2 \\ y_1 ' & y_2 '
    \end{bmatrix} =  \begin{bmatrix} \alpha ' \\ \beta ' \end{bmatrix}
    = \begin{bmatrix}  0 \\ r(t)    \end{bmatrix} \]
  
  \[  \alpha  ' = \frac{\det \begin{bmatrix} 0 && y_2 \\ r && y_2' \end{bmatrix}}
      {\det \begin{bmatrix} y_1 & y_2 \\ y_1 ' & y_2 ' \end{bmatrix}} = -
      \frac{r(t) y_2 (t) }{y_1 y_2' - y_2 y_1 } = - \frac{r(t) y_2(t) }{W[y_1, y_2](t)} \]
    \[ \beta  ' = \frac{\det \begin{bmatrix} y_1 & 0 \\ y_1 ' & r(t) \end{bmatrix} }
    {\det \begin{bmatrix} y_1 & y_2 \\ y_1 ' & y_2 ' \end{bmatrix}} = -
    \frac{r(t) y_1 (t) }{y_1 y_2' - y_2 y_1 } = - \frac{r(t) y_1(t) }{W[y_1, y_2](t)} \]

Integrando as soluções temos \( \alpha , \beta \) , após isso substituimos em \( y_P(t) = \alpha (t)
y_1(t) + \beta (t) y_2(t)\)  e a
solução geral da EDO será :

\[ y(t) = \left( \alpha _0 + \alpha (t) \right) y_1(t) + \left( \beta _0 + \beta (t) \right)y_2(t) \]

 \paragraph{$\blacksquare$ Exemplo}: \( x^2 y '' - 3x y' + 4y = x^2 \ln |x| , y(x) = ?, x> 0 \) 

 Considerando a EDO homogênea associada temos
 \[ x^2 y'' - 3xy' + 4y = 0  \]
 
 Suponto \( y(x) = x^{\lambda } \Rightarrow y' = \lambda x^{\lambda -1} \Rightarrow y'' = \lambda
 (\lambda - 1) x^{\lambda - 2}  \) 
 
 \[ x^{\lambda } \left[ \lambda (\lambda - 1) - 3 \lambda  + 4  \right] = 0 \Rightarrow \lambda
   ^2 - 4 \lambda  + 4 = 0 \Rightarrow \left( \lambda - 2 \right)^2 = 0 \Rightarrow \lambda  = 2 \]
 Logo, \( \lambda = 2 \) e uma das soluções da EDO homogenea associada é \( y_1 (x) = x^2 \) 


 Usando método do Wronskiano para encontrar \( y_2(t) \)

 \[ y_1 y_2 ' - y_1 ' y_2 = W_0 e^{- \int p(x) dx}, p(x) = - \frac{3}{x} \]
 \[ x^2 y_2 ' - 2x y_2 = e^{3 \ln x} = e^{x^3}  \]
 \[ y_2 ' - \frac{2}{x} y_2 = x \Rightarrow \mu = \frac{1}{x^2} \]
 \[ \left( \mu y_2 \right)' = \frac{1}{x} \Rightarrow \frac{y_2}{x^2} = \ln x + \mathbb{C}
   \Rightarrow y_2(x) = x^2 \ln x + x^2 \mathbb{C}\]
 Fazendo \( \mathbb{C} = 0 \), temos que \( y_2(x) = x^2 \ln x \) 
 uma possível solução é \( y_2(x) = x^2 \ln x \)   e a outra é \( y_1(x) = x^2 \) 
 Para encontrar \( y_P(x) \) impomos:
 \[ y_P(x) = \alpha (x) y_1(x) + \beta (x) y_2 (x) = \alpha (x) x^2 + \beta (x) x^2 \ln x\]
 \[ y_P'(x) =  \alpha ' x^2 + 2 \alpha  x + \beta ' x^2 \ln x + \beta (2x\lnx + x) \]
 impondo também \( \alpha ' x^2 + \beta ' x^2 \ln x = 0  \) 
 
 \[ y_P '   =  2 \alpha x + \beta (2x \ln x + x) \]
 \[ y_P'' = 2 \alpha ' x + 2 \alpha + \beta ' (2x \ln x + x ) + \beta (2 \ln x + 3) \]
 Substituindo na EDO:
 \[ y_P '' - \frac{3}{x} y_P ' + \frac{4}{x^2} y_P = \ln x \]
 \begin{multline*}
   \Leftrightarrow \left[ 2 \alpha ' x + 2 \alpha  + \beta ' (2x \ln x + x) + \beta (2 \ln x + 3)
   \right] - \frac{3}{x} \left[ 2 \alpha x + \beta \left( 2x \ln x + x \right) \right]+
   \\ \frac{4}{x^2} \left( \alpha x^2 + \beta x^2 \ln x  \right) = \ln x \end{multline*}
 \[ 2 \alpha ' x + \left( 2x \ln x + x \right)\beta ' = \ln x \]
 Temos então o sistema
 $$\begin{cases}
     x^2 \alpha ' + 2 x \ln x \beta ' = 0 \\
     2 \alpha ' x + \left( 2x \ln x + x \right) \beta ' = \ln x  
   \end{cases}$$
   
   \[ \Leftrightarrow \begin{cases}
          \alpha ' + \ln x \beta ' = 0 \\ 2 \alpha ' + \left( 2 \ln x + 1\right) \beta ' = \frac{\ln x}{x}
        \end{cases} \]

      \[ \alpha ' = - \beta ' \ln x   \]
      \[ - \beta ' \ln x + 2\ln x \beta' + \beta ' = \frac{\ln x}{x} \]
      \[ \beta ' = \frac{\ln x }{x} \Rightarrow \alpha ' = - \frac{(\ln x)^2}{x} \]

      \[ \beta = \int \frac{\ln x}{x} dx \text{ e } \alpha = - \int \frac{(\ln x)^2}{x} dx \]
      \[ \beta = \frac{(\ln x)^2}{2} \text{ e } \alpha = - \frac{(\ln x )^3}{3} \]

      \[ y_P(x) = - \frac{(\ln x )^3 x^2}{3} + \frac{(\ln x)^2 x^2 \ln x}{2}\]
      \[ y_P(x) = \frac{x^2 (\ln x)^3}{6} \]

      A solução geral é:

      \[ y(x) =  \alpha _0 x^2 + \beta _0 x^2 \ln x + \frac{x^2 (\ln x)^3 }{6}\]
      onde \( \alpha_0 \)   e \( \beta_0 \) são as condições de contorno.

      

      \[ a(t) y''(t) + b(t) y'(t) + c(t) y(t) = f(t) \]
      
      Solução geral: \( y(t) = y_{H} (t) + y_{P} (t) \) 

      onde \( y_H \) é uma solução geral da EDO homogênea associada (obtida substituindo
      f(t) por zero e resolvendo a EDO homogênea resultante) e \( y_P \) é uma solução particular
      qualquer da EDO inomogênea dada.
      
      Supondo que já resolvemos a EDO homogênea por algum método:
      
      \[ y_H(t) = \alpha  y_1 (t) + \beta y_2(t) \]
      onde já conhecemos as funções \( y_1 \) e \( y_2 \) .
      
      
      Método da variação de parâmetros para encontrar \( y_P(t) \) 

      Reescrevendo a solução particular da inomogênea como :

      \[ y_P(t) = \alpha (t) y_1(t) + \beta (t) y_2(t) \]
      
   E colocando a EDO original na forma canônica:
   \[ y_P' = \alpha y_1 ' + \alpha ' y_1 + \beta  y_2 ' + \beta ' y_2 \]

   Impomos a relação:
   \begin{equation}
   \alpha ' y_1 + \beta ' y_2 = 0  
   \label{eq:inomogeneas_1}
   \end{equation}
   segue que \( y_P ' = \alpha (t) y_1 '(t) + \beta (t) y_2 ' (t) \) 


   \[ y_P '' =  \alpha ' y_1 ' + \alpha y_1 ''  + \beta ' y_2 ' + \beta y_2 '' \]


   Substituindo na EDO dada:

   \[ y_P '' + p y_P ' + q y_P  = \alpha y_1 '' + \alpha ' y_1 ' + \beta y_2 '' + \beta ' y_2 ' + p
   \left( \alpha y_1 ' + \beta  y_2 '  \right) + q \left( \alpha y_1 + \beta  y_2 \right) = r(t) \]


   \[ \alpha (t) \underbrace{\left[ y_1 '' (t) + p(t) y_1' (t)  + q(t) y_1(t) \right]}_{ =  0}  + \beta (t) \underbrace{\left[
   y_2''(t) + p(t) y_2 ' (t) + q(t) y_2 (t)  \right]}_{ = 0 } + \alpha ' y_1' + \beta ' y_2' = r(t) \]

   \begin{equation}
   \alpha ' y_1 ' + \beta ' y_2 ' = r(t) 
   \label{inomogeneas_2}
   \end{equation}


   Com as duas equações (\ref{inomogeneas_1}) e (\ref{eq:inomogeneas_2}) temos o sistema linear:

   \[ \begin{cases}
   y_1 \alpha ' + y_2 \beta ' = 0 \\
   y_1 ' \alpha ' + y_2 ' \beta ' = r(t)   
    \end{cases}\]

   \[ \begin{bmatrix}
   y_1 & y_2 \\ y_1 ' & y_2 '
   \end{bmatrix} =  \begin{bmatrix} \alpha ' \\ \beta ' \end{bmatrix}
= \begin{bmatrix}  0 \\ r(t)    \end{bmatrix} \]

\[  \alpha  ' = \frac{\det \begin{bmatrix} 0 && y_2 \\ r && y_2' \end{bmatrix} }{\det \begin{bmatrix}
y_1 & y_2 \\ y_1 ' & y_2 ' \end{bmatrix}} = -
\frac{r(t) y_2 (t) }{y_1 y_2' - y_2 y_1 } = - \frac{r(t) y_2(t) }{W[y_1, y_2](t)} \]

\[  \beta  ' = \frac{\det \begin{bmatrix} y_1 & 0 \\ y_1 ' & r(t) \end{bmatrix} }{\det \begin{bmatrix}
y_1 & y_2 \\ y_1 ' & y_2 ' \end{bmatrix}} = -
\frac{r(t) y_1 (t) }{y_1 y_2' - y_2 y_1 } = - \frac{r(t) y_1(t) }{W[y_1, y_2](t)} \]


Integrando as soluções temos \( \alpha , \beta \) , após isso substituimos em \( y_P(t) = \alpha (t)
y_1(t) + \beta (t) y_2(t)\)  e a
solução geral da EDO será :


\begin{equation}
  y(t) = \left( \alpha _0 + \alpha (t) \right) y_1(t) + \left( \beta _0 + \beta (t) \right)y_2(t) 
  \label{eq:solucao_geral_inomogenea}
\end{equation}

 \paragraph{$\blacksquare$ Exemplo} : \( x^2 y '' - 3x y' + 4y = x^2 \ln |x| , y(x) = ?, x> 0 \) 

 Considerando a EDO homogênea associada temos
 \[ x^2 y'' - 3xy' + 4y = 0  \]

 Supondo \( y(x) = x^{\lambda } \Rightarrow y' = \lambda x^{\lambda -1} \Rightarrow y'' = \lambda
 (\lambda - 1) x^{\lambda - 2}  \) 
 
 \[ x^{\lambda } \left[ \lambda (\lambda - 1) - 3 \lambda  + 4  \right] = 0 \Rightarrow \lambda
   ^2 - 4 \lambda  + 4 = 0 \Rightarrow \left( \lambda - 2 \right)^2 = 0 \Rightarrow \lambda  = 2 \]
 Logo, \( \lambda = 2 \) e uma das soluções da EDO homogenea associada é \( y_1 (x) = x^2 \) 
 
 
 Usando método do Wronskiano para encontrar \( y_2(t) \)
 
 \[ y_1 y_2 ' - y_1 ' y_2 = W_0 e^{- \int p(x) dx}, p(x) = - \frac{3}{x} \]
 \[ x^2 y_2 ' - 2x y_2 = e^{3 \ln x} = e^{x^3}  \]
 \[ y_2 ' - \frac{2}{x} y_2 = x \Rightarrow \mu = \frac{1}{x^2} \]
 \[ \left( \mu y_2 \right)' = \frac{1}{x} \Rightarrow \frac{y_2}{x^2} = \ln x + \mathbb{C}
   \Rightarrow y_2(x) = x^2 \ln x + x^2 \mathbb{C}\]
 Fazendo \( \mathbb{C} = 0 \), temos que \( y_2(x) = x^2 \ln x \) 
 
 uma possível solução é \( y_2(x) = x^2 \ln x \)   e a outra é \( y_1(x) = x^2 \) 
 
 
 Para encontrar \( y_P(x) \) impomos:
 
 \[ y_P(x) = \alpha (x) y_1(x) + \beta (x) y_2 (x) = \alpha (x) x^2 + \beta (x) x^2 \ln x\]
 \[ y_P'(x) =  \alpha ' x^2 + 2 \alpha  x + \beta ' x^2 \ln x + \beta (2x\lnx + x) \]
 
 
 impondo também \( \alpha ' x^2 + \beta ' x^2 \ln x = 0  \) 
 
 \[ y_P '   =  2 \alpha x + \beta (2x \ln x + x) \]
 \[ y_P'' = 2 \alpha ' x + 2 \alpha + \beta ' (2x \ln x + x ) + \beta (2 \ln x + 3) \]
 
 Substituindo na EDO:
 \[ y_P '' - \frac{3}{x} y_P ' + \frac{4}{x^2} y_P = \ln x \]
 \begin{multline*}
   \Leftrightarrow \left[ 2 \alpha ' x + 2 \alpha  + \beta ' (2x \ln x + x) + \beta (2 \ln x + 3)
   \right] - \frac{3}{x} \left[ 2 \alpha x + \beta \left( 2x \ln x + x \right) \right]+
   \\ \frac{4}{x^2} \left( \alpha x^2 + \beta x^2 \ln x  \right) = \ln x \end{multline*}
 
 \[ 2 \alpha ' x + \left( 2x \ln x + x \right)\beta ' = \ln x \]
 
 
 
 Temos então o sistema
 
 $$\begin{cases}
     x^2 \alpha ' + 2 x \ln x \beta ' = 0 \\
     2 \alpha ' x + \left( 2x \ln x + x \right) \beta ' = \ln x  
   \end{cases}$$
   
   \[ \Leftrightarrow \begin{cases}
          \alpha ' + \ln x \beta ' = 0 \\ 2 \alpha ' + \left( 2 \ln x + 1\right) \beta ' = \frac{\ln x}{x}
        \end{cases} \]
      
      \[ \alpha ' = - \beta ' \ln x   \]
      \[ - \beta ' \ln x + 2\ln x \beta' + \beta ' = \frac{\ln x}{x} \]
      \[ \beta ' = \frac{\ln x }{x} \Rightarrow \alpha ' = - \frac{(\ln x)^2}{x} \]
      
      
      
      \[ \beta = \int \frac{\ln x}{x} dx \text{ e } \alpha = - \int \frac{(\ln x)^2}{x} dx \]
      
      \[ \beta = \frac{(\ln x)^2}{2} \text{ e } \alpha = - \frac{(\ln x )^3}{3} \]
      
      
      \[ y_P(x) = - \frac{(\ln x )^3 x^2}{3} + \frac{(\ln x)^2 x^2 \ln x}{2}\]
      \[ y_P(x) = \frac{x^2 (\ln x)^3}{6} \]
      
   
      
      A solução geral é:

      \[ y(x) =  \alpha _0 x^2 + \beta _0 x^2 \ln x + \frac{x^2 (\ln x)^3 }{6}\]
      onde \( \alpha_0 \)   e \( \beta_0 \) são as condições de contorno.



%%% Local Variables:
%%% mode: latex
%%% coding: utf-8
%%% TeX-master: "main.tex"
%%% End:


